% Options for packages loaded elsewhere
\PassOptionsToPackage{unicode}{hyperref}
\PassOptionsToPackage{hyphens}{url}
%
\documentclass[
]{article}
\usepackage{amsmath,amssymb}
\usepackage{lmodern}
\usepackage{iftex}
\ifPDFTeX
  \usepackage[T1]{fontenc}
  \usepackage[utf8]{inputenc}
  \usepackage{textcomp} % provide euro and other symbols
\else % if luatex or xetex
  \usepackage{unicode-math}
  \defaultfontfeatures{Scale=MatchLowercase}
  \defaultfontfeatures[\rmfamily]{Ligatures=TeX,Scale=1}
\fi
% Use upquote if available, for straight quotes in verbatim environments
\IfFileExists{upquote.sty}{\usepackage{upquote}}{}
\IfFileExists{microtype.sty}{% use microtype if available
  \usepackage[]{microtype}
  \UseMicrotypeSet[protrusion]{basicmath} % disable protrusion for tt fonts
}{}
\makeatletter
\@ifundefined{KOMAClassName}{% if non-KOMA class
  \IfFileExists{parskip.sty}{%
    \usepackage{parskip}
  }{% else
    \setlength{\parindent}{0pt}
    \setlength{\parskip}{6pt plus 2pt minus 1pt}}
}{% if KOMA class
  \KOMAoptions{parskip=half}}
\makeatother
\usepackage{xcolor}
\usepackage[margin=1in]{geometry}
\usepackage{color}
\usepackage{fancyvrb}
\newcommand{\VerbBar}{|}
\newcommand{\VERB}{\Verb[commandchars=\\\{\}]}
\DefineVerbatimEnvironment{Highlighting}{Verbatim}{commandchars=\\\{\}}
% Add ',fontsize=\small' for more characters per line
\usepackage{framed}
\definecolor{shadecolor}{RGB}{248,248,248}
\newenvironment{Shaded}{\begin{snugshade}}{\end{snugshade}}
\newcommand{\AlertTok}[1]{\textcolor[rgb]{0.94,0.16,0.16}{#1}}
\newcommand{\AnnotationTok}[1]{\textcolor[rgb]{0.56,0.35,0.01}{\textbf{\textit{#1}}}}
\newcommand{\AttributeTok}[1]{\textcolor[rgb]{0.77,0.63,0.00}{#1}}
\newcommand{\BaseNTok}[1]{\textcolor[rgb]{0.00,0.00,0.81}{#1}}
\newcommand{\BuiltInTok}[1]{#1}
\newcommand{\CharTok}[1]{\textcolor[rgb]{0.31,0.60,0.02}{#1}}
\newcommand{\CommentTok}[1]{\textcolor[rgb]{0.56,0.35,0.01}{\textit{#1}}}
\newcommand{\CommentVarTok}[1]{\textcolor[rgb]{0.56,0.35,0.01}{\textbf{\textit{#1}}}}
\newcommand{\ConstantTok}[1]{\textcolor[rgb]{0.00,0.00,0.00}{#1}}
\newcommand{\ControlFlowTok}[1]{\textcolor[rgb]{0.13,0.29,0.53}{\textbf{#1}}}
\newcommand{\DataTypeTok}[1]{\textcolor[rgb]{0.13,0.29,0.53}{#1}}
\newcommand{\DecValTok}[1]{\textcolor[rgb]{0.00,0.00,0.81}{#1}}
\newcommand{\DocumentationTok}[1]{\textcolor[rgb]{0.56,0.35,0.01}{\textbf{\textit{#1}}}}
\newcommand{\ErrorTok}[1]{\textcolor[rgb]{0.64,0.00,0.00}{\textbf{#1}}}
\newcommand{\ExtensionTok}[1]{#1}
\newcommand{\FloatTok}[1]{\textcolor[rgb]{0.00,0.00,0.81}{#1}}
\newcommand{\FunctionTok}[1]{\textcolor[rgb]{0.00,0.00,0.00}{#1}}
\newcommand{\ImportTok}[1]{#1}
\newcommand{\InformationTok}[1]{\textcolor[rgb]{0.56,0.35,0.01}{\textbf{\textit{#1}}}}
\newcommand{\KeywordTok}[1]{\textcolor[rgb]{0.13,0.29,0.53}{\textbf{#1}}}
\newcommand{\NormalTok}[1]{#1}
\newcommand{\OperatorTok}[1]{\textcolor[rgb]{0.81,0.36,0.00}{\textbf{#1}}}
\newcommand{\OtherTok}[1]{\textcolor[rgb]{0.56,0.35,0.01}{#1}}
\newcommand{\PreprocessorTok}[1]{\textcolor[rgb]{0.56,0.35,0.01}{\textit{#1}}}
\newcommand{\RegionMarkerTok}[1]{#1}
\newcommand{\SpecialCharTok}[1]{\textcolor[rgb]{0.00,0.00,0.00}{#1}}
\newcommand{\SpecialStringTok}[1]{\textcolor[rgb]{0.31,0.60,0.02}{#1}}
\newcommand{\StringTok}[1]{\textcolor[rgb]{0.31,0.60,0.02}{#1}}
\newcommand{\VariableTok}[1]{\textcolor[rgb]{0.00,0.00,0.00}{#1}}
\newcommand{\VerbatimStringTok}[1]{\textcolor[rgb]{0.31,0.60,0.02}{#1}}
\newcommand{\WarningTok}[1]{\textcolor[rgb]{0.56,0.35,0.01}{\textbf{\textit{#1}}}}
\usepackage{graphicx}
\makeatletter
\def\maxwidth{\ifdim\Gin@nat@width>\linewidth\linewidth\else\Gin@nat@width\fi}
\def\maxheight{\ifdim\Gin@nat@height>\textheight\textheight\else\Gin@nat@height\fi}
\makeatother
% Scale images if necessary, so that they will not overflow the page
% margins by default, and it is still possible to overwrite the defaults
% using explicit options in \includegraphics[width, height, ...]{}
\setkeys{Gin}{width=\maxwidth,height=\maxheight,keepaspectratio}
% Set default figure placement to htbp
\makeatletter
\def\fps@figure{htbp}
\makeatother
\setlength{\emergencystretch}{3em} % prevent overfull lines
\providecommand{\tightlist}{%
  \setlength{\itemsep}{0pt}\setlength{\parskip}{0pt}}
\setcounter{secnumdepth}{-\maxdimen} % remove section numbering
\ifLuaTeX
  \usepackage{selnolig}  % disable illegal ligatures
\fi
\IfFileExists{bookmark.sty}{\usepackage{bookmark}}{\usepackage{hyperref}}
\IfFileExists{xurl.sty}{\usepackage{xurl}}{} % add URL line breaks if available
\urlstyle{same} % disable monospaced font for URLs
\hypersetup{
  pdftitle={LAB 25 (MD)},
  pdfauthor={AnaGSanjuanM},
  hidelinks,
  pdfcreator={LaTeX via pandoc}}

\title{LAB 25 (MD)}
\author{AnaGSanjuanM}
\date{2023-02-23}

\begin{document}
\maketitle

---------------LABORATORIO 25---------------------

Tidy data - datos ordenados - Parte 1

Objetivo: Introducción práctica a los datos ordenados (o tidy data) y a
las herrameintas que provee el paquete tidyr.

En este ejecicio vamos a:

\begin{enumerate}
\def\labelenumi{\arabic{enumi}.}
\item
  Cargar datos (tibbles)
\item
  Convesrtir nuestros tibbles en datadrames (para su exportación)
\item
  Exportar dataframes originales
\item
  Pivotar tabla 4a
\item
  Exportar resultado (TABLA PIVOTANTE)
\end{enumerate}

Prerrequisitos. instalar paquete tidyverse

install.packages(``tidyverse'')

Instalar paquete de datos

install.packages(``remotes'')

remotes::install\_github(``cienciadedatos/datos'')

install.packages(``datos'')

Cargar paquete tidyverse

\begin{Shaded}
\begin{Highlighting}[]
\FunctionTok{library}\NormalTok{(tidyverse)}
\end{Highlighting}
\end{Shaded}

\begin{verbatim}
## -- Attaching core tidyverse packages ------------------------ tidyverse 2.0.0 --
## v dplyr     1.1.0     v readr     2.1.4
## v forcats   1.0.0     v stringr   1.5.0
## v ggplot2   3.4.1     v tibble    3.1.8
## v lubridate 1.9.2     v tidyr     1.3.0
## v purrr     1.0.1     
## -- Conflicts ------------------------------------------ tidyverse_conflicts() --
## x dplyr::filter() masks stats::filter()
## x dplyr::lag()    masks stats::lag()
## i Use the ]8;;http://conflicted.r-lib.org/conflicted package]8;; to force all conflicts to become errors
\end{verbatim}

Cargar paquete de datos

\begin{Shaded}
\begin{Highlighting}[]
\FunctionTok{library}\NormalTok{(}\StringTok{"datos"}\NormalTok{)}
\end{Highlighting}
\end{Shaded}

Tabla 1 hasta tabla 4b. Ver datos como tibble

\begin{Shaded}
\begin{Highlighting}[]
\NormalTok{datos}\SpecialCharTok{::}\NormalTok{tabla1}
\end{Highlighting}
\end{Shaded}

\begin{verbatim}
## # A tibble: 6 x 4
##   pais        anio  casos  poblacion
##   <chr>      <dbl>  <dbl>      <dbl>
## 1 Afganistán  1999    745   19987071
## 2 Afganistán  2000   2666   20595360
## 3 Brasil      1999  37737  172006362
## 4 Brasil      2000  80488  174504898
## 5 China       1999 212258 1272915272
## 6 China       2000 213766 1280428583
\end{verbatim}

Cargar las otras cuatro tablas del paquete de datos

\begin{Shaded}
\begin{Highlighting}[]
\NormalTok{datos}\SpecialCharTok{::}\NormalTok{tabla2}
\end{Highlighting}
\end{Shaded}

\begin{verbatim}
## # A tibble: 12 x 4
##    pais        anio tipo          cuenta
##    <chr>      <dbl> <chr>          <dbl>
##  1 Afganistán  1999 casos            745
##  2 Afganistán  1999 población   19987071
##  3 Afganistán  2000 casos           2666
##  4 Afganistán  2000 población   20595360
##  5 Brasil      1999 casos          37737
##  6 Brasil      1999 población  172006362
##  7 Brasil      2000 casos          80488
##  8 Brasil      2000 población  174504898
##  9 China       1999 casos         212258
## 10 China       1999 población 1272915272
## 11 China       2000 casos         213766
## 12 China       2000 población 1280428583
\end{verbatim}

\begin{Shaded}
\begin{Highlighting}[]
\NormalTok{datos}\SpecialCharTok{::}\NormalTok{tabla3}
\end{Highlighting}
\end{Shaded}

\begin{verbatim}
## # A tibble: 6 x 3
##   pais        anio tasa             
##   <chr>      <dbl> <chr>            
## 1 Afganistán  1999 745/19987071     
## 2 Afganistán  2000 2666/20595360    
## 3 Brasil      1999 37737/172006362  
## 4 Brasil      2000 80488/174504898  
## 5 China       1999 212258/1272915272
## 6 China       2000 213766/1280428583
\end{verbatim}

\begin{Shaded}
\begin{Highlighting}[]
\NormalTok{datos}\SpecialCharTok{::}\NormalTok{tabla4a}
\end{Highlighting}
\end{Shaded}

\begin{verbatim}
## # A tibble: 3 x 3
##   pais       `1999` `2000`
##   <chr>       <dbl>  <dbl>
## 1 Afganistán    745   2666
## 2 Brasil      37737  80488
## 3 China      212258 213766
\end{verbatim}

\begin{Shaded}
\begin{Highlighting}[]
\NormalTok{datos}\SpecialCharTok{::}\NormalTok{tabla4b}
\end{Highlighting}
\end{Shaded}

\begin{verbatim}
## # A tibble: 3 x 3
##   pais           `1999`     `2000`
##   <chr>           <dbl>      <dbl>
## 1 Afganistán   19987071   20595360
## 2 Brasil      172006362  174504898
## 3 China      1272915272 1280428583
\end{verbatim}

La visualización de los datos es como tibble, es como un dataframe pero
no se puede exportar, cambiar variables (hay ciertas limitantes).

Convirtiendo tibble a dataframe

Primer datadrame (df1) que sea igual a la importación de un data\_frame
de la tabla1.

\begin{Shaded}
\begin{Highlighting}[]
\NormalTok{df1 }\OtherTok{\textless{}{-}} \FunctionTok{data\_frame}\NormalTok{(tabla1)}
\end{Highlighting}
\end{Shaded}

\begin{verbatim}
## Warning: `data_frame()` was deprecated in tibble 1.1.0.
## i Please use `tibble()` instead.
\end{verbatim}

Ahora ya aparece en el Environment y se puede exportar

Se ejecuta la misma indicación con las tablas restantes, asigando nombre
de dataframe con base en la tabla.

\begin{Shaded}
\begin{Highlighting}[]
\NormalTok{df2 }\OtherTok{\textless{}{-}} \FunctionTok{data\_frame}\NormalTok{(tabla2)}
\NormalTok{df3 }\OtherTok{\textless{}{-}} \FunctionTok{data\_frame}\NormalTok{(tabla3)}
\NormalTok{df4a }\OtherTok{\textless{}{-}} \FunctionTok{data\_frame}\NormalTok{(tabla4a)}
\NormalTok{df4b }\OtherTok{\textless{}{-}} \FunctionTok{data\_frame}\NormalTok{(tabla4b)}
\end{Highlighting}
\end{Shaded}

Para visualizar

\begin{Shaded}
\begin{Highlighting}[]
\NormalTok{df1}
\end{Highlighting}
\end{Shaded}

\begin{verbatim}
## # A tibble: 6 x 4
##   pais        anio  casos  poblacion
##   <chr>      <dbl>  <dbl>      <dbl>
## 1 Afganistán  1999    745   19987071
## 2 Afganistán  2000   2666   20595360
## 3 Brasil      1999  37737  172006362
## 4 Brasil      2000  80488  174504898
## 5 China       1999 212258 1272915272
## 6 China       2000 213766 1280428583
\end{verbatim}

\begin{Shaded}
\begin{Highlighting}[]
\NormalTok{df2}
\end{Highlighting}
\end{Shaded}

\begin{verbatim}
## # A tibble: 12 x 4
##    pais        anio tipo          cuenta
##    <chr>      <dbl> <chr>          <dbl>
##  1 Afganistán  1999 casos            745
##  2 Afganistán  1999 población   19987071
##  3 Afganistán  2000 casos           2666
##  4 Afganistán  2000 población   20595360
##  5 Brasil      1999 casos          37737
##  6 Brasil      1999 población  172006362
##  7 Brasil      2000 casos          80488
##  8 Brasil      2000 población  174504898
##  9 China       1999 casos         212258
## 10 China       1999 población 1272915272
## 11 China       2000 casos         213766
## 12 China       2000 población 1280428583
\end{verbatim}

\begin{Shaded}
\begin{Highlighting}[]
\NormalTok{df3}
\end{Highlighting}
\end{Shaded}

\begin{verbatim}
## # A tibble: 6 x 3
##   pais        anio tasa             
##   <chr>      <dbl> <chr>            
## 1 Afganistán  1999 745/19987071     
## 2 Afganistán  2000 2666/20595360    
## 3 Brasil      1999 37737/172006362  
## 4 Brasil      2000 80488/174504898  
## 5 China       1999 212258/1272915272
## 6 China       2000 213766/1280428583
\end{verbatim}

\begin{Shaded}
\begin{Highlighting}[]
\NormalTok{df4a}
\end{Highlighting}
\end{Shaded}

\begin{verbatim}
## # A tibble: 3 x 3
##   pais       `1999` `2000`
##   <chr>       <dbl>  <dbl>
## 1 Afganistán    745   2666
## 2 Brasil      37737  80488
## 3 China      212258 213766
\end{verbatim}

\begin{Shaded}
\begin{Highlighting}[]
\NormalTok{df4b}
\end{Highlighting}
\end{Shaded}

\begin{verbatim}
## # A tibble: 3 x 3
##   pais           `1999`     `2000`
##   <chr>           <dbl>      <dbl>
## 1 Afganistán   19987071   20595360
## 2 Brasil      172006362  174504898
## 3 China      1272915272 1280428583
\end{verbatim}

¿Cuál de las bases está ordenada y cuál está desordenada?

Los datos irdenados tienen tres características esenciales

\begin{enumerate}
\def\labelenumi{\arabic{enumi}.}
\item
  Cada variable debe contener su propia columna
\item
  Cada observación debe tener su propia fila
\item
  Cada valor debe tener su propia celda
\end{enumerate}

Exportar los dataframes originales

\begin{Shaded}
\begin{Highlighting}[]
\FunctionTok{write.csv}\NormalTok{(df1,}\AttributeTok{file=}\StringTok{"df1.csv"}\NormalTok{)}
\end{Highlighting}
\end{Shaded}

Se hace lo mismo con los otros dataframes

\begin{Shaded}
\begin{Highlighting}[]
\FunctionTok{write.csv}\NormalTok{(df2,}\AttributeTok{file=}\StringTok{"df2.csv"}\NormalTok{)}
\FunctionTok{write.csv}\NormalTok{(df3,}\AttributeTok{file=}\StringTok{"df3.csv"}\NormalTok{)}
\FunctionTok{write.csv}\NormalTok{(df4a,}\AttributeTok{file=}\StringTok{"df4a.csv"}\NormalTok{)}
\FunctionTok{write.csv}\NormalTok{(df4b,}\AttributeTok{file=}\StringTok{"df4b.csv"}\NormalTok{)}
\end{Highlighting}
\end{Shaded}

Ordenando los datos.

Explicación de tibble en cuadrante de visualizaciones

\begin{Shaded}
\begin{Highlighting}[]
\FunctionTok{vignette}\NormalTok{(}\StringTok{"tibble"}\NormalTok{)}
\end{Highlighting}
\end{Shaded}

\begin{verbatim}
## starting httpd help server ... done
\end{verbatim}

La mayoría de las funciones que usarás en este libro, producen tibbles,
ya que son una de las caracterísitcas transversales del tidyverse.

Si ya te has familiarizado con data.frame(), es importante que tomes en
cuanta que tibble() hace menos cosas.

Nunca cambia el tipo de los inputs (p.~ej., ¡nunca convierte caracteres
en factores!)

Nunca cambia el nombre de las variables y nunca asigna nombres a las
filas.

Ordenar datos con la tabla4a (PIVOTAR). Se añade el operador pipe
\%\textgreater\% (presionando ctrl Shift M)

Se genera objeto llamado t4a\_PIVOTANTE (será una tabla ordenada), para
pivotear a lo largo (pivot\_longer)

Las columnas están dadas por los años y se reemplazará el nombre por
anio en los que englobará los dos momentos en el tiempo.

Los valores se tomarán como casos.

\begin{Shaded}
\begin{Highlighting}[]
\NormalTok{t4a\_PIVOTANTE }\OtherTok{=}\NormalTok{ tabla4a }\SpecialCharTok{\%\textgreater{}\%}
  \FunctionTok{pivot\_longer}\NormalTok{(}\AttributeTok{cols=}\FunctionTok{c}\NormalTok{(}\StringTok{"1999"}\NormalTok{, }\StringTok{"2000"}\NormalTok{), }\AttributeTok{names\_to=}\StringTok{"anio"}\NormalTok{, }\AttributeTok{values\_to =} \StringTok{"casos"}\NormalTok{)}
\end{Highlighting}
\end{Shaded}

Para visualizar

\begin{Shaded}
\begin{Highlighting}[]
\NormalTok{t4a\_PIVOTANTE}
\end{Highlighting}
\end{Shaded}

\begin{verbatim}
## # A tibble: 6 x 3
##   pais       anio   casos
##   <chr>      <chr>  <dbl>
## 1 Afganistán 1999     745
## 2 Afganistán 2000    2666
## 3 Brasil     1999   37737
## 4 Brasil     2000   80488
## 5 China      1999  212258
## 6 China      2000  213766
\end{verbatim}

Exportar resultados: tabla ordenada

\begin{Shaded}
\begin{Highlighting}[]
\FunctionTok{write.csv}\NormalTok{(t4a\_PIVOTANTE, }\AttributeTok{file=}\StringTok{"t4a\_PIVOTANTE.csv"}\NormalTok{)}
\end{Highlighting}
\end{Shaded}

------------------------FIN DE LABORATORIO
25-----------------------------------

\end{document}
